
\documentclass[11pt]{article}

% Packages
\usepackage[czech]{babel}
\usepackage[utf8]{inputenc}
\usepackage[useregional]{datetime2}
\usepackage[T1]{fontenc}
\usepackage[a4paper, total={15.24cm, 23.32cm}]{geometry}
\usepackage[thinlines]{easytable}
\usepackage{graphicx}
\usepackage[ampersand]{easylist}
\usepackage{changepage}
\usepackage{float}
\usepackage{color}
\usepackage{enumitem}
\usepackage{hyperref}

% Config
%\setlength\parindent{0pt}
\renewcommand{\baselinestretch}{1.2} 
\setitemize{itemsep=0pt}
\hypersetup{
	colorlinks,
	citecolor=black,
	filecolor=black,
	linkcolor=black,
	urlcolor=black
}
\title{\textbf{V. Počítačová grafika a analýza obrazu}}
\date{\small\vspace{-9ex}Update: \today}

\begin{document}
\maketitle
\setcounter{tocdepth}{1}
\tableofcontents
\pagebreak

\section{Osvětlovací modely a systémy barev v počítačové grafice.}

\pagebreak
\section{Afinní a projektivní prostor. Afinní a projektivní transformace a jejich matematický zápis. Aplikace v počítačové grafice. Modelovací a zobrazovací transformace.}

\pagebreak
\section{Křivky a plochy: teoretické základy (definice, rovnice, tečný a normálový vektor, křivosti, Cn a Gn spojitost), použití (Bézier, Coons, NURBS).}

\pagebreak
\section{Geometrické a objemové modelování. Hraniční metoda, metoda CSG, výčet prostoru, oktantové stromy.}

\pagebreak
\section{Standardní zobrazovací řetězec a realizace jeho jednotlivých kroků. Gouraudovo a Phongovo stínování. Řešení viditelnosti. Grafický standard OpenGL: stručná charakteristika.}

\pagebreak
\section{Metody získávání fotorealistických obrázků (rekurzivní sledování paprsku, vyzařovací metoda, renderovací rovnice).}

\pagebreak
\section{Komprese obrazu a videa; principy úprav obrazu v prostorové a frekvenční doméně.}

\pagebreak
\section{Základní metody úpravy a segmentace obrazu (filtrace, prahování, hrany).}

\pagebreak
\section{Základní metody rozpoznávání objektů (příznakové rozpoznávání).}

\end{document}
