\documentclass[11pt]{article}

% Packages
\usepackage[czech]{babel}
\usepackage[utf8]{inputenc}
\usepackage[useregional]{datetime2}
\usepackage[T1]{fontenc}
\usepackage[a4paper, total={15.24cm, 23.32cm}]{geometry}
\usepackage[thinlines]{easytable}
\usepackage{graphicx}
\usepackage[ampersand]{easylist}
\usepackage{changepage}
\usepackage{float}
\usepackage{color}
\usepackage{enumitem}

% Config
\setlength\parindent{0pt}
\renewcommand{\baselinestretch}{1.2} 
\setitemize{itemsep=0pt}

\title{\textbf{III. Databázové a informační systémy}}
\date{\small\vspace{-9ex}Update: \today}

\begin{document}
\maketitle

\section{Modelování databázových systémů, konceptuální modelování, datová analýza, funkční analýza; nástroje a modely.}

\pagebreak
\section{Relační datový model, SQL; funkční závislosti, dekompozice a normální formy.}

\pagebreak
\section{Transakce, zotavení, log, ACID, operace COMMIT a ROLLBACK; problémy souběhu, řízení souběhu: zamykání, úroveň izolace v SQL.}

\pagebreak
\section{Procedurální rozšíření SQL, PL/SQL, T-SQL, triggery, funkce, procedury, kurzory, hromadné operace.}

\pagebreak
\section{Základní fyzická implementace databázových systémů: tabulky a indexy; plán vykonávání dotazů.}

\pagebreak
\section{Objektově‐relační datový model a XML datový model: principy, dotazovací jazyky.}

\pagebreak
\section{Datová vrstva informačního systému; existující API, rámce a implementace, bezpečnost; objektově-relační mapování.}

\pagebreak
\section{Distribuované SŘBD, fragmentace a replikace.}

\end{document}
