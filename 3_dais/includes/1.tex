\subsection*{Konceptuální model}
Konceptuální model je jednoduchý popis entit a jejich vzájemných vztahů používaný v databázích. Jedná se o jakýsi prvotní jednoduchý návrh námi vytvářené databáze. Je kladen důraz na zobrazeních všech entit a jejich vztahů. Nezávislý na SŘBD.

\subsubsection*{Základní pojmy}
\begin{itemize}
\item \textbf{Entita} - objekt reálného světa.
\item \textbf{Atribut} - vlastnost entity (možné hodnoty jsou označeny jako doména atributu)
\item \textbf{Entitní typ} - množina entit se stejnými atributy.
\item \textbf{Vztah} - vztah mezi \textbf{2 entitními} typy.
\item \textbf{Popis} entitních typů, jejich atributů a vztahů mezi nimi v rámci daného zadání.
\item \textbf{E-R diagram} - grafické znázornění konceptuálního modelu, neexistuje standard (používá se např. crow foot notace).
\item \textbf{Kardinalita vztahu} - Dělení vztahů podle počtu entit vstupujících do vztahu: 
\begin{itemize}
\item 1:1
\item 1:N
\item M:N
\end{itemize}
\end{itemize}

\subsection*{Funkční analýza}
\begin{itemize}
\item Cílem je popsat vytvářený systém jako „černou skříňku“, definovat její \textbf{vnější chování} a strukturalizovat \textbf{okolí systému}, které se systémem komunikuje. \textbf{Popsat všechny funkce, které se budou s daty provádět.}
\item \textbf{Vnitřní:} rozpracované algoritmy pro jednotlivé akce.
\item \textbf{Vnější:} náhled na strukturu a hierarchii funkcí.
\item Úkolem je získat od zadavatele\textbf{ seznam všech funkcí}, které bude od IS požadovat.
\item Sestaví se tabulka se sloupci událost a reakce, do ní se formou krátkých textů zapisují všechny možné vnější události, podněty působící na systém a jim odpovídající reakce systému. 
\end{itemize}

\subsubsection*{Otázky na požadavky}
\begin{itemize}
\item\textbf{PROČ} nový systém. 
\item\textbf{ČEMU} má sloužit. 
\item\textbf{KDO} s ním pracuje - běžně, příležitostně, pravidelně zřídka. 
\item\textbf{VSTUPY} – objekty, atributy 
\item\textbf{VÝSTUPY} – výstupní sestavy, požadované informace 
\item\textbf{FUNKCE} – jaké výpočty, odvozování, výběry, třídění, \ldots
\item\textbf{Vazby na OKOLÍ systému} – odkud data a kam.
\end{itemize}

\subsubsection*{Nefunkční požadavky}
\begin{itemize}
\item Požadavky na \textbf{výsledný program}
\item \textbf{Vnější požadavky}: ostatní nefunkční implementační požadavky, použití \textbf{standardů}, \textbf{cenová} omezení, \textbf{časové} požadavky.
\end{itemize}

\subsubsection*{Diagram datových toků (DFD)}
\begin{itemize}
\item Několik úrovní podrobnosti.
\item Definuje\textbf{ hranici systému}.
\item Definuje \textbf{všechny akce}, které mezi systémem a jeho okolím probíhají.
\end{itemize}

\subsubsection*{Minispecifikace = algoritmy elementárních funkcí}
\begin{itemize}
\item Pro každou nerozložitelnou funkci z DFD \textbf{existuje minispecifikace}.
\item Popisuje postup, jak jsou vstupní \textbf{data transformována na výstupní}.
\item Popisuje, co \textbf{funkce znamená} věcně, ne, jak se to spočítá.
\item Používá se \textbf{přirozený jazyk} s omezeným množstvím jasně definovaných pojmů.
\item Musí být \textbf{srozumitelná} analytikovi, uživateli i programátorovi.
\end{itemize}


\subsection*{Datová analýza}
\begin{itemize}
\item Zabývá strukturou obsahové části systému (strukturou databaze).
\item Datový model, obsahuje již samotné atributy - \textbf{Závislý na SŘBD.}
\item \textbf{Popis struktury dat IS}, jejich \textbf{vazby}, uchovávané vlastnosti a možné hodnoty těchto vlastností.
\item \textbf{Využívá se}: lineární zápis entit a atributů, ER diagram, {Konceptuální} schéma, {datový slovník} (tabulky atributů).
\end{itemize}


\subsection*{Problémy agendového zpracování}
\begin{itemize}
\item\textbf{Redundance}: některé informace ve více souborech opakují, jsou redundantní. Redundance je zdrojem mnoha dalších problémů.
\item\textbf{Konzistence}: vzájemná \textbf{shoda údajů}. Postupem času - vlivem nedostatečné kontroly v programech se stejné hodnoty na různých místech v datových souborech, začnou rozcházet.
\item\textbf{Integrita}: data aktuální, odrážejí skutečnost z reálného světa. Problémem tedy je zabezpečit, aby chybou či nedůsledností uživatele nebyla porušena integrita a konzistence dat.
\item\textbf{Izolovanost dat}:  data roztroušena v různých souborech, soubory mohou být různě organizovány, data různě formátována. To komplikuje tvorbu nových aplikačních programů a možnost realizovat vazby mezi datovými strukturami.
\item\textbf{Současný přístup více uživatelů}: větší systémy vyžadují současný přístup k datům více uživatelů. Pak je nutné, aby programy vzájemně spolupracovaly, jejich činnosti byly koordinovány.
\item\textbf{Ochrana proti zneužití}: při zpracování důvěrných či tajných dat není přípustné, aby měl kdokoliv přístup ke všem informacím. Při klasickém zpracování však musí mít programátor aplikačních programů k dispozici tolik podrobností, že to ochranu dat prakticky znemožňuje.
\end{itemize}
