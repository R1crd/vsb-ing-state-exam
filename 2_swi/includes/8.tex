\subsection{Jazyk UML}
\textbf{UML je jazyk umožňující specifikaci, vizualizaci, konstrukci a dokumentaci artefaktů softwarového systému}. V průběhu let se UML stal \textbf{standardizovaným jazykem }určeným pro vytvoření výkresové dokumentace (softwarového) systému. K vytváření jednotlivých modelů systému jazyk UML poskytuje \textbf{celou řadu diagramů} umožňujících \textbf{postihnout různé aspekty systému}.  Jedná se celkem o čtyři základní náhledy a k nim přiřazené diagramy: 

\begin{enumerate}
\item \textbf{Funkční náhled}
\begin{enumerate}
\item \textbf{Diagram případů užití} - popisující vztahy mezi aktéry a jednotlivými případy použití. 
\end{enumerate}
\item \textbf{Logický náhled}
\begin{enumerate}
\item \textbf{Diagram tříd} - specifikující množinu tříd, rozhraní a jejich vzájemné vztahy. Tyto diagramy slouží k vyjádření statického pohledu na systém.
\item \textbf{Objektový diagram}
\end{enumerate}
\item \textbf{Dynamický náhled popisující chování}
\begin{enumerate}
\item \textbf{Stavový diagram} - dokumentující životní cyklus objektu dané třídy z hlediska jeho stavů, přechodů mezi těmito stavy a událostmi, které tyto přechody uskutečňují. 
\item \textbf{Diagram aktivit} -  popisující podnikový proces pomocí jeho stavů reprezentovaných vykonáváním aktivit a pomocí přechodů mezi těmito stavy způsobených ukončením těchto aktivit. Účelem diagramu aktivit je blíže popsat tok činností daný vnitřním
mechanismem jejich provádění. 
\item \textbf{Interakční diagramy}
\begin{enumerate}
\item \textbf{Sekvenční diagramy} -  popisující interkace mezi objekty z hlediska jejich časového uspořádaní.
\item \textbf{Diagramy spolupráce} -  je obdobně jako předchozí sekvenční diagram zaměřen na interkace, ale z pohledu strukturální organizace objektů. Jinými slovy není primárním aspektem časová posloupnost posílaných zpráv, ale \textbf{topologie rozmístění objektů}. 
\end{enumerate}
\end{enumerate}
\item \textbf{Implementační náhled}
\begin{enumerate}
\item \textbf{Diagram komponent} - ilustrující organizaci a závislosti mezi softwarovými komponentami. 
\item \textbf{Diagram nasazení} - upřesněný nejen ve smyslu konfigurace technických prostředků, ale především z hlediska rozmístění implementovaných softwarových komponent na těchto prostředcích.
\end{enumerate}
\end{enumerate}

\subsection{Diagramy a jejich použití v rámci fází vývoje}
\begin{itemize}
\item \textbf{Specifikace požadavků}: Diagram případů užití, Sekvenční diagramy, Diagram aktivit.
\item \textbf{Návrh}: Diagram tříd, Objektový diagram, Stavový diagram, Sekvenční diagramy.
\item \textbf{Implementace}: Diagramy spolupráce, Diagram komponent, Diagram nasazení.
\end{itemize}
