\documentclass[11pt]{article}

% Packages
\usepackage[czech]{babel}
\usepackage[utf8]{inputenc}
\usepackage[useregional]{datetime2}
\usepackage[T1]{fontenc}
\usepackage[a4paper, total={15.24cm, 23.32cm}]{geometry}
\usepackage[thinlines]{easytable}
\usepackage{graphicx}
\usepackage[ampersand]{easylist}
\usepackage{changepage}
\usepackage{float}
\usepackage{color}
\usepackage{enumitem}

% Config
\setlength\parindent{0pt}
\renewcommand{\baselinestretch}{1.2} 
\setitemize{itemsep=0pt}

\title{\textbf{I. Matematické základy informatiky}}
\date{\small\vspace{-9ex}Update: \today}

\begin{document}
\maketitle

\section{Konečné automaty, regulární výrazy, uzávěrové vlastnosti třídy regulárních jazyků.}

\pagebreak
\section{Bezkontextové gramatiky a jazyky. Zásobníkové automaty, jejich vztah k bezkontextovým gramatikám.}

\pagebreak
\section{Matematické modely algoritmů -Turingovy stroje a stroje RAM. Složitost algoritmu, asymptotické odhady. Algoritmicky nerozhodnutelné problémy.}

\pagebreak
\section{Třídy složitosti problémů. Třída PTIME a NPTIME, NP-úplné problémy.}

\pagebreak
\section{Jazyk predikátové logiky prvního řádu. Práce s kvantifikátory a ekvivalentní transformace formulí.}

\pagebreak
\section{Pojem relace, operace s relacemi, vlastnosti relací. Typy binárních relací. Relace ekvivalence a relace uspořádání.}

\pagebreak
\section{Pojem operace a obecný pojem algebra. Algebry s jednou a dvěma binárními operacemi.}

\pagebreak
\section{FCA – formální kontext, formální koncept, konceptuální svazy. Asociační pravidla, hledání často se opakujících množin položek.}

\pagebreak
\section{Metrické a topologické prostory – metriky a podobnosti.}

\pagebreak
\section{Shlukování.}

\pagebreak
\section{Náhodná veličina. Základní typy náhodných veličin. Funkce určující rozdělení náhodných veličin.}

\pagebreak
\section{Vybraná rozdělení diskrétní a spojité náhodné veličiny - binomické, hypergeometrické, negativně binomické, Poissonovo, exponenciální, Weibullovo, normální rozdělení.}

\pagebreak
\section{Popisná statistika. Číselné charakteristiky a vizualizace kategoriálních a kvantitativních proměnných.}

\pagebreak
\section{Metody statistické indukce. Intervalové odhady. Princip testování hypotéz. Okruhy pokývají předměty Teoretická informatika, Pravděpodobnost a statistika, Matematika pro zpracování znalostí}

\end{document}
