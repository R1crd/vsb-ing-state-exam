\subsection{Monolitické počítače (mikroprocesory)}
\begin{itemize}
\item Mikroprocesory, mikrokontroléry, minipočítače jsou další názvy pro monolitické počíače.
\item Jsou to \textbf{malé počítače integrované v jediném pouzdře} (all in one) s nízkým výkonem (oproti běžným CPU).
\item Mají širokou oblast využití.
\item Využívá se Harvardské koncepce, což umožňuje aplikovat paměti pro data a program různých technologií.
\item Zjednodušené rysy architektury RISC.
\item INTEL 8051 (standard), ATMEL, MICROCHIP PIC.
\item V monolitických počítačích můžeme najít dva základní typy periférií (\textbf{vstupní}/\textbf{výstupní}).
\item Často umožňují se přepnout celý MCU do režimu spánku, a tím výrazně snížit svou spotřebu. Poté běží pouze malá část periferií a ostatní jsou odpojeny.
\end{itemize}
 
 \subsubsection{Rozdělení paměti}
\begin{itemize}
		\item \textbf{Pro data} -- používáme většinou paměti energeticky závislé typu \textbf{RWM--RAM} (\textit{Read--Write Memory--Random Access Memory}), tedy paměť s libovolným přístupem pro čtení i zápis. Jsou vyráběny jako statické (uchování paměti po celou dobu napájení), jejich paměťové buňky jsou realizovány jako klopné obvody. 
		\item \textbf{Pro program} -- se používají paměti typu \textbf{ROM} (\textit{Read--Only Memory}) určené především ke čtění (paměť je uchována i po odpojení napájení). Mezi nejčastěji používané paměti patří \textbf{EPROM}, \textbf{EEPROM} (\textit{Electrically Erasable Programmable Read-Only Memory}), \textbf{PROM} (\textit{Programmable Read Only Memory}) a \textbf{Flash}.
\end{itemize}

\subsection{Organizace paměti}
\begin{itemize}
	\item \textbf{Střadačové (pracovní) registry }-- ve struktuře procesoru jsou obvykle \textbf{1-8-16} základních pracovních registrů, jsou nejpoužívanější. Ukládají se do nich \textbf{aktuálně zpracovávaná data} a jsou nejčastějším operandem strojových instrukcí (to na co se instrukce v závorkách odkazují). A také se do nich nejčastěji ukládají výsledky operací. Nejsou určeny pro dlouhodobé ukládání dat. Nejrychlejší.
	\item \textbf{Univerzální zápisníkové registry} -- jsou jich desítky až stovky. Slouží pro ukládání \textbf{nejčastěji používaných dat}. Instrukční soubor obvykle dovoluje, aby se část strojových instrukcí prováděla přímo s těmito registry. Formát strojových instrukcí ovšem obvykle nedovoluje adresovat velký rozsah registru, proto se implementuje několik stejných skupin registru vedle sebe, s možností mezi skupinami přepínat - \textbf{registrové banky}.
	\item \textbf{Paměť dat RWM} -- slouží pro ukládání \textbf{rozsáhlejších} nebo \textbf{méně používaných dat} (z těch předešlých nejméně používaný). Instrukční soubor obvykle nedovoluje s obsahem této paměti přímo manipulovat, kromě instrukcí přesunových. Těmi se data přesunou např. do pracovního registru. Některé procesory dovolují, aby data z této paměti byla použita jako druhý operand strojové instrukce, výsledek ale nelze zpět do této paměti uložit přímo. Nejpomalejší.
\end{itemize}

\subsection{Zdroje synchronizace}
MCU umožňuje využít interní oscilátory nebo na jeden vstup přivést externí oscilátor a tím tak určit přesný takt procesoru. Možnosti zdrojů jsou:
\begin{itemize}
	\item \textbf{krystal} (křemenný výbrus) – jsou drahé ale přesné,
	\item keramický rezonátor,
	\item obvod RC -- snadno integrovatelný,
	\item obvod LC -- méně časté.
\end{itemize}

\subsection{Ochrana proti rušení}
Na prvním místě jde o ochranu \textbf{mechanickou}. Odolávat náhodným nárazům, nebo i trvalým vibracím nebo \textbf{elemtromagnetickým} vlivům z okolí. Pro odstranění chyb, které nastanou působením vnějších vlivů nebo chyby programátora, je v mikropočítačích implementován speciální obvod nazývaný \textbf{WATCHDOG} $\,\to\,$. Jedná se o interní čítač, který musí být programem pravidelně resetován. Pokud nedojde k resetu watchdogu a ten přeteče, je celý MCU RESETován. Nastává nejčastěji při zacyklení. Watchod (WDT) se řadí mezi elektrické ochrany. Tam také můžeme zařadit \textbf{BROWN-OUT} -- ochrana proti podpěti.

\subsection{Typické periferie}
\textbf{Periferie} - obvody, které zajišťují komunikaci mikropočítace s okolím.

\begin{enumerate}
\item \textbf{Vstupní a výstupní brány} -- Nejjednodušší a nejčastěji používané rozhraní pro vstup a výstup informací je u mikropočítačů \textbf{paralelní brána - port}. Bývá obvykle organizována jako  \textbf{x jednobitových vývodů}, kde lze současně zapisovat i číst logické informace 0 a 1. Nejčastější jsou 8 bitové porty, existují ale také 4, 6, 7 bitové i další. U většiny bran lze jednotlivě \textbf{nastavit}, které bitové vývody budou sloužit jako \textbf{vstupní} a které jako \textbf{výstupní} a v průběhu programu tyto vlastnosti měnit. Na vstupu je \textbf{Schmittův klopný obvod}. U mnoha mikropočítačů jsou brány implementovány tak, že s nimi instrukční soubor může pracovat jako s množinou vývodu, nebo jako s jednotlivými bity.

\item \textbf{Čítace a časovače} -- Do skupiny nejpoužívanějších periférií mikropočítače určitě patří čítače a časovače.
\emph{Časovač} se od \textbf{čítače} příliš neliší. Není, ale \textbf{inkrementován} vnějším signálem, ale \textbf{přímo vnitřním hodinovým signálem} používaným pro řízení samotného mikropočítače. Lze tak podle přesnosti zdroje hodinového signálu zajistit řízení událostí a chování v \textbf{reálném čase}. Při přetečení časovače se i zde může automaticky předávat signál do přerušovacího podsystému mikropočítače.

Ve většině MCU je čítač i časovač implementován jako 1 periferie a lze programově nastavit, je li řízena vnějším či vnitřním hodinovým signálem. Navíc lze také programově zařazovat děličky, čímž lze snížit celkovou frekvenci inkrementace.

\item \textbf{Sériové linky} -- Sériový přenos dat je v praxi stále více používán. Dovoluje efektivním způsobem
přenášet data na relativně velké vzdálenosti při použití minimálního počtu vodičů. Hlavní nevýhodou je však \textbf{nižší přenosová rychlost}, a to že se data musí kódovat a dekódovat.
\begin{itemize}
\item \textbf{USART} (\textbf{RS232}) +/-12V jet transformována na TTL/RS422/RS485.
\item \textbf{I2C} (Philips) komunikace mezi integrovanými obvody (přenos dat uvnitř elektronického zařízení).
\item \textbf{SPI}
\end{itemize}

\item \textbf{A/D a D/A převodníky} -- Fyzikální veličiny, které vstupují do mikropočítače, jsou většinou reprezentovány
\textbf{analogovou formou} (napětím, proudem, nebo odporem).Pro zpracování počítačem však potřebujeme informaci v digitální (číselné) formě. K tomuto účelu slouží analogově–číslicové převodníky.

A/D převodník je nejčastěji realizován komparátorem, kde se postupně zvyšuje napětí a porovnává se se vstupní hodnotou. Tím se určí výsledná úroveň, která má ale určité rozlišení. Je omezený počet kroků, ve kterých se hodnota porovnává. Tento převod ale trvá nějakou dobu a nelze tak meřit velmi rychle se měnící signál.

D/A převodník většinou v MCU neexistuje. Pro převod digitálního signálu na analogový se používá PWM (pulsně šířková modulace). Tedy velmi rychle se měnící signál plného a žádného napětí, který je následně kondenzátorem vyhlazen.

\item \textbf{Obvody reálného času} (RTC - Real Time Clock) - V mnoha aplikacích s použítím mikropočítačů je potřeba dodržovat přesnou časovou souvislost řízených událostí. Jde tedy o řízení v reálném čase. Ne vždy, ale taková posloupnost dostačuje a je nutno pro potřebu řízení udržovat skutečný čas, tedy hodiny, minuty, sekundy a případně i zlomky sekund. Pro tyto účely slouží obvody \textbf{RTC}. Při jejich použití je obvykle nutné vyřešit dva základní problémy:
\begin{itemize}
\item \textbf{záložní zdroj} -- je třeba zajistit záložní zdroj pro udržení nepřetržité činnosti obvodu (může dojít k výpadku proudu a tak i k ztrátě skutečného času).
\item \textbf{čtení dat} -- čas je hodnota neustále se měnící. Např. pokud zahájíme čtení hodnoty v čase 10:59:59, může se stát, že po přečtění prvních dvou hodnot, v našem případě hodin, se čas posune na 11:00:00 a čtění dalších hodnot bude neplatné (řešení technicky pomocnými registry v RTC obvodu, nebo vhodným programovým řešením). 
\end{itemize}
\end{enumerate}

\subsection{I$^2$C}
\textbf{Multi-masterová počítačová sériová sběrnice} vyvinutá firmou Philips, která je používána k připojování nízkorychlostních periferií k základní desce nebo vestavěnému systému.
\begin{itemize}
\item Dvoudrátová, dvouvodičová sběrnice se sériovým přenosem.
\item Obsahuje \textbf{master} (zahajuje a ukončuje komunikaci; generuje hodinový signál SCL) a \textbf{slave} (zařízení adresované masterem) obvody.
\item Lze propojit až 128 zařízení (Master, slave).
\item \textbf{Adresa zařízení} -- skládá se ze 7 bitů (horní 4 určuje výrobce, dolní 3 jdou nastavit libovolně).
\item \textbf{Signály} -- SCL (synchronous clock), SDA (synchronous data).
\end{itemize}